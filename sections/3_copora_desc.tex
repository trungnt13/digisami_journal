\section{The Corpora and their Annotations}
\label{sec:corpora-annotations}

\subsection{The North Sami DigiSami Corpus}
\label{sec:digisami-corpus}

The DigiSami Corpus of spoken North Sami was collected in the areas traditionally inhabited by the Sami people: in Enonteki\"{o}, Utsjoki, Inari and Ivalo in Finland, and in Kautokeino and Karasjok in Norway.
%(see the map in Figure~\ref{fig:data-locations}).
North Sami belongs to the Fenno-Ugric language family, and is one of the Sami languages
spoken in Northern Scandinavia, Finland and the Kola Peninsula \cite{Hirvonen:ea:97}.
%\cite{Seurujarvi-Kari:ea:97}.
All the speakers are native speakers of North Sami, and their ages vary between 16 and 65 years.
The data is thus versatile, including informants from two different countries and of different ages.
More details about the data collection can be found in \cite{Jokinen:Wilcock:SLTU:14} and \cite{Jokinen:LREC:14}).

The participants were invited to take part in three different tasks: discussion and writing Wikipedia articles, reading aloud of existing Wikipedia texts, and taking part in a free conversation which was video recorded. However, in the context of studying laughing and engagement, only the conversations are considered.
The conversations were recorded by EDIROL R4Pro four-channel recording device with AKG 417 L-microphones. Two Panasonic HC-X920 video cameras and three GoPro HERO3 cameras were used for video-recordings. Conversational speech was also recorded by the cameras own microphone. The conversations were between two or three people, and the participants were instructed to discuss freely about their own interests or about the Wikipedia articles they were to write (e.g. Sami language, Sami costume, music, reindeer herding, and snow-mobiles). The topics vary from everyday life (next vacation, driving school, cars) to translation between Sami and other languages and to technological tools that have been made to help writing North Sami more correctly. The conversations differ in style: familiar participants have casual conversations and they often refer to things they had been talking about earlier. The conversations between a pupil and a teacher are fairly formal, and the topics stick to the forthcoming task, i.e. things that one could write a Wikipedia article about.

The corpus consists of more than three hours (195 minutes) of annotated data; the mean duration of the conversations was 10:07 minutes, and it is valuable because it is the first North Sami conversational corpus. The total number of laughter occurrences in the North Sami data is 341 in 8 different conversations. Two of these conversations were recorded in Karasjok, Norway, and the rest in Ivalo and Utsjoki, Finland. There were 19 conversation informants altogether. 11 of the participants were female and 8 were male. Altogether, 59\% (201) of the laughter occurrences were performed by a female informant, while 41\% (140) were performed by a male informant. Table~\ref{tab:laughter-digisami} shows the number of different laughter types in different conversations.

\subsection{The Estonian MINT Corpus}
\label{sec:estonian-mint-corpus}

The Estonian First Encounters data was collected in the project MINT \cite{Jokinen:Tenjes:12}. The MINT corpus of first encounters follows the guidelines of the project NOMCO \cite{Paggio:ea:10}. The first encounter dialogues engage participants, who do not know each other in advance, in an activity where their task is to chat and make acquaintance with each other. Original data was collected in the Estonian language, and the data is annotated and analysed using an annotation scheme which is co-measurable with the annotations used in NOMCO. Each participant was given a short presentation of the project and the goals of the data collection before the recording, and they were also asked to sign a consent form (in the Estonian language) that grants permission for their video data to be used for research purposes, and to be shown to third parties without further permission.

SonyHDR-XR550V cameras with three external Sony ECM-HW2 wireless microphones were used in the recordings. The video recording was on the full HD quality mode, and the camera views were cut, edit and merged via Sony Vegas Pro 11.
% See Figure 2.

There were a total of 23 participants (12 male and 11 female), with age ranging between 21 and 61 years. The participants are native speakers of Estonian and they are students or university employees. All conversations were two-participant dialogues, and each participant had two conversations with two different partners. The interactions were gender balanced so that there were 8 female-female encounters, 7 female-male encounters, and 8 male-male encounters. The corpus contains 23 conversations, and the mean duration of the interaction was 5:52 minutes.

None of the participants knew each other beforehand, as opposed to the DigiSami data, in which all participants knew each other or were even close friends. In most conversations, the topics were related to introductory and self-presentation issues, mainly studies or work.

The laughter occurrences in the Estonian MINT Corpus were annotated as described in Table~\ref{tab:laughter-types}. The total number of laughter occurrences in the Estonian MINT data was 519 in 23 different conversations.
%All conversations had 2 participants and the mean duration of the conversations was 5:52 minutes. 22 of the participants were female and 24 male - each participant attended 2 different conversations with a different partner (one participant had 2 different partners).
Altogether, 67\% (348) of the laughter occurrences were performed by a female informant, while 33\% (171) were performed by a male informant.

\subsection{The Finnish NOMCO Corpus}
\label{sec:finnish-nomco-corpus}
